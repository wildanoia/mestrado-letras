\documentclass[
article,			% Define que o documento é um artigo
12pt,				% Tamanho da fonte
oneside,			% Impressão apenas em um lado do papel
a4paper,			% Tamanho do papel
english,			% Idioma adicional para hifenização
brazil				% Idioma principal
]{abntex2}

\usepackage[utf8]{inputenc}
\usepackage[brazil]{babel}
\usepackage[T1]{fontenc}
\usepackage{lipsum}     % Gera texto de preenchimento

% --- PACOTE DE CITAÇÕES (Faltava este) ---
\usepackage[alf]{abntex2cite}	% Citações sistema Autor-Data (Padrão Letras/Humanas)
% -----------------------------------------

% --- DADOS DO ARTIGO ---
% Nota: Em artigos curtos, às vezes não usamos \maketitle do abntex2 
% pois ele cria uma capa inteira. Abaixo faremos um cabeçalho manual.

\begin{document}
	
	% --- CABEÇALHO MANUAL (Melhor para artigos) ---
	\begin{center}
		\textbf{\large TÍTULO DO ARTIGO: SUBTÍTULO SE HOUVER} \\
		\vspace{1.5cm}
		Seu Nome\footnote{Mestrando em Letras - UFT. E-mail: seuemail@uft.edu.br} \\
		\small{Universidade Federal do Tocantins (UFT) - Porto Nacional}
	\end{center}
	\vspace{1cm}
	% ----------------------------------------------
	
	% Resumo em Português
	\begin{resumo}
		Este é o resumo do artigo. O resumo deve ressaltar o objetivo, o método, os resultados e as conclusões do documento. A ordem e a extensão destes itens dependem do tipo de resumo (informativo ou indicativo) e do tratamento que cada item recebe no documento original.
		
		\vspace{\onelineskip}
		\noindent
		\textbf{Palavras-chave}: Literatura. Linguística. UFT.
	\end{resumo}
	
	% (Opcional) Resumo em Inglês - Abstract
	% \begin{resumo}[Abstract]
		%  \begin{otherlanguage*}{english}
			%    This is the abstract in English.
			%
			%    \vspace{\onelineskip}
			%    \noindent
			%    \textbf{Keywords}: Literature. Linguistics. UFT.
			%  \end{otherlanguage*}
		% \end{resumo}
	
	\section{Introdução}
	Aqui você começa a escrever sua introdução. Exemplo de citação indireta \cite{authoryear}.
	\lipsum[1] % Texto de exemplo
	
	\section{Desenvolvimento}
	Aqui entra a discussão teórica.
	\lipsum[2-3]
	
	\section{Conclusão}
	\lipsum[4]
	
	% --- REFERÊNCIAS ---
	% Lembre-se de criar o arquivo referencias.bib na mesma pasta
	\bibliography{referencias}
	
\end{document}