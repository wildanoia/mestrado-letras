% --- CONFIGURAÇÕES INICIAIS ---
\documentclass{beamer}

% Tema e Codificação
\usetheme{Madrid}
\usepackage[utf8]{inputenc}
\usepackage[brazil]{babel} 

% --- DEFINIÇÃO DE CORES UFT (Extraídas do Manual) ---
% Definindo os nomes das cores baseados nos valores RGB da imagem
\definecolor{UFTAzul}{RGB}{0, 84, 132}      % Azul Institucional
\definecolor{UFTAmarelo}{RGB}{253, 185, 19} % Amarelo Institucional
\definecolor{UFTVerde}{RGB}{0, 133, 119}    % Verde Institucional
\definecolor{UFTCinza}{RGB}{102, 102, 102}  % Cinza Institucional

% --- APLICANDO AS CORES AO TEMA MADRID ---
% Muda a cor dos títulos, ícones de lista e estrutura geral para o Azul
\setbeamercolor{structure}{fg=UFTAzul}

% Configura as cores das caixas (blocos) e rodapés
% Palette Primary: Usado na parte direita do rodapé (número da página) e títulos
\setbeamercolor{palette primary}{bg=UFTAzul, fg=white}

% Palette Secondary: Usado na parte do meio do rodapé (Título curto)
\setbeamercolor{palette secondary}{bg=UFTVerde, fg=white}

% Palette Tertiary: Usado na parte esquerda do rodapé (Autor)
\setbeamercolor{palette tertiary}{bg=UFTCinza, fg=white}

% Configura a cor de texto em destaque (\alert{}) para o Amarelo
\setbeamercolor{alerted text}{fg=UFTAmarelo}

% Remove os ícones de navegação pequenos (opcional, deixa o slide mais limpo)
\setbeamertemplate{navigation symbols}{}

% --- DADOS DA APRESENTAÇÃO ---
\title[PPG-Letras UFT]{Título do Seminário}
\subtitle{Disciplina: Nome da Disciplina}
\author{Seu Nome}
\institute[UFT]{
	Universidade Federal do Tocantins \\
	PPG em Letras - Porto Nacional
}
\date{\today}

% --- INÍCIO DOS SLIDES ---
\begin{document}
	
	% --- SLIDE 1: CAPA ---
	\begin{frame}
		\titlepage
	\end{frame}
	
	% --- SLIDE 2: SUMÁRIO ---
	\begin{frame}{Sumário}
		\tableofcontents
	\end{frame}
	
	% --- SEÇÃO 1 ---
	\section{Introdução}
	
	\begin{frame}{Introdução com Cores da UFT}
		Observe como as cores oficiais foram aplicadas:
		\vspace{0.5cm}
		\begin{itemize}
			\item Este marcador (bolinha) está no \textbf{Azul UFT}.
			\item O título do slide está no \textbf{Azul UFT}.
			\item \alert{Este texto está em destaque (Amarelo UFT).}
		\end{itemize}
		
		\vspace{0.5cm}
		% Exemplo de bloco colorido
		\begin{block}{Exemplo de Bloco}
			O tema Madrid usa as cores definidas para criar estes blocos automaticamente.
		\end{block}
	\end{frame}
	
	% --- SEÇÃO 2 ---
	\section{Metodologia}
	
	\begin{frame}{Metodologia}
		Descrição da metodologia aplicada.
		\begin{enumerate}
			\item Passo um
			\item Passo dois
		\end{enumerate}
	\end{frame}
	
	% --- SEÇÃO 3 ---
	\section{Resultados}
	
	\begin{frame}{Resultados}
		Discussão sobre os textos lidos.
	\end{frame}
	
\end{document}